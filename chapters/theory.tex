\chapter{Propuesta}

\section{Consideraciones Iniciales}
En la presente tesis se propone un enfoque innovador para la detección de fases sísmicas utilizando modelos basados en Transformers, específicamente EQTransformer. Este modelo ha demostrado ser efectivo en la clasificación de eventos sísmicos y se adapta bien a las limitaciones de recursos computacionales, lo que lo hace adecuado para aplicaciones en tiempo real y sistemas de alerta temprana.

El objetivo principal es evaluar el rendimiento de EQTransformer en la detección de ondas P y S, comparándolo con otros modelos basados en Transformers y enfoques tradicionales. Se busca determinar la viabilidad de estos modelos para su implementación en sistemas de monitoreo sísmico, especialmente en entornos con recursos limitados. Y ademas proponer mejoras en la arquitectura y el entrenamiento del modelo para optimizar su rendimiento. Para lograr esto, se utilizarán conjuntos de datos de sismogramas etiquetados, y se realizarán experimentos para medir la precisión, la recuperación y la puntuación F1 de los modelos en la detección de fases sísmicas. La principal contribución de esta tesis es demostrar que los modelos basados en Transformers, como EQTransformer, pueden superar a los enfoques tradicionales en la detección de ondas P y S, ofreciendo una alternativa robusta y eficiente para el análisis sísmico. Sin embargo la principal meta de esta tesis es agregar un nuevo modelo basado en Transformers que sea capaz de detectar las ondas P y S, y que sea capaz de ser ejecutado en dispositivos embebidos, como lo es el modelo ICAT-net.

\section{Esquema de la Propuesta}

\section{Esquema de la Propuesta}

La innovación del marco propuesto en esta tesis radica en la integración del módulo de Atención por Coordenadas (Coordinate Attention) con el mecanismo de atención de Transformers, y la implementación de una estructura de modelo híbrido que combina Transformers y redes convolucionales a través de un módulo de concatenación, como se muestra en la Figura 4. 

A través de la función del módulo de Atención por Coordenadas (Ecuaciones \ref{eq:conv1d} y \ref{eq:sigmoid}), las características de entrada se transforman en características mejoradas $Z$, haciendo que las características clave en los datos originales sean más prominentes. El objetivo es resaltar la información crucial dentro de la matriz de características de entrada $X$. 

En arquitecturas de aprendizaje profundo, las conexiones residuales son una técnica ampliamente utilizada que permite la transferencia directa de información entre diferentes capas. Esta técnica es particularmente importante para la construcción de redes profundas, ya que ayuda a mitigar los problemas de gradientes que desaparecen o explotan, permitiendo así el entrenamiento de redes más profundas.

\begin{equation}
Y = \text{Conv1d}(X; \text{kernel size} = 1, \text{stride} = 1)
\label{eq:conv1d}
\end{equation}

\begin{equation}
Z = \sigma(Y)
\label{eq:sigmoid}
\end{equation}

Donde $\sigma(Y)$ representa la función Sigmoide. La característica mejorada $Z$ se divide posteriormente en tres flujos de procesamiento independientes. Un flujo procesa a través de una red neuronal convolucional unidimensional con parámetros $\text{kernel size} = 1$ y $\text{stride} = 1$ para calcular el valor de consulta $Q$ para el Transformer, como se describe en la Ecuación \ref{eq:query}.

\begin{equation}
Q = \text{Conv1d}(Z; \text{kernel size} = 1, \text{stride} = 1)
\label{eq:query}
\end{equation}

En otro flujo de procesamiento, los datos se someten a un submuestreo y se reducen dimensionalmente a través de las ecuaciones \ref{eq:downsampling}, \ref{eq:key}, y \ref{eq:value} para derivar las claves $K$ y los valores $V$ para el mecanismo de atención, un proceso destinado a reducir la complejidad computacional.

\begin{equation}
Z' = \text{DownSampling}(Z)
\label{eq:downsampling}
\end{equation}

\begin{equation}
K = \text{Conv1d}(Z'; \text{kernel size} = 1, \text{stride} = 1)
\label{eq:key}
\end{equation}

\begin{equation}
V = \text{Conv1d}(Z'; \text{kernel size} = 1, \text{stride} = 1)
\label{eq:value}
\end{equation}

\section{Dataset}
El dataset utilizado en esta tesis es el mismo que el utilizado por EQTransformer, que consiste en sismogramas etiquetados con las fases P y S. Este dataset contiene una variedad de eventos sísmicos, lo que permite entrenar y evaluar los modelos de manera efectiva. La diversidad de los datos es crucial para garantizar que los modelos aprendan a generalizar y no se sobreajusten a características específicas de un subconjunto de datos. El nombre del dataset es "STEAD", que contiene registros de eventos sísmicos de diferentes magnitudes y ubicaciones geográficas. Este dataset es ampliamente utilizado en la comunidad de investigación sísmica y proporciona una base sólida para el entrenamiento y la evaluación de modelos de detección de fases sísmicas.

\section{Entrenamiento del Modelo}

El entrenamiento del modelo se realiza utilizando el dataset STEAD, que contiene registros de eventos sísmicos con etiquetas de fases P y S. El proceso de entrenamiento implica la optimización de los parámetros del modelo para minimizar la función de pérdida, que en este caso es la entropía cruzada entre las predicciones del modelo y las etiquetas reales. Se utiliza un optimizador Adam con una tasa de aprendizaje adaptativa para ajustar los pesos del modelo durante el entrenamiento.

\section{Resultados}

Los resultados del modelo se evalúan utilizando métricas estándar como precisión, recuperación y puntuación F1. Estas métricas permiten medir la efectividad del modelo en la detección de fases sísmicas P y S. Los resultados obtenidos muestran que el modelo EQTransformer supera a los enfoques tradicionales en términos de precisión y recuperación, lo que indica su capacidad para detectar fases sísmicas de manera más efectiva. Primeramente se mostrara netamente los resultados obtenidos por EQTransformer, y posteriormente se comparara con otros modelos basados en Transformers y enfoques tradicionales.

\section{Consideraciones Finales}

En conclusión, la propuesta de utilizar modelos basados en Transformers, específicamente EQTransformer, para la detección de fases sísmicas P y S ha demostrado ser efectiva y prometedora. La integración del módulo de Atención por Coordenadas y la estructura híbrida del modelo permiten una mejor captura de las características relevantes en los sismogramas, mejorando así la precisión y recuperación en la detección de fases sísmicas.