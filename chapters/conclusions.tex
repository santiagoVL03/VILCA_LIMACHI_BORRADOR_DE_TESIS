\chapter{Conclusiones y trabajos futuros}\label{conclusions}

\section{Conclusiones}
En esta tesis, se ha presentado un enfoque innovador para la detección de fases sísmicas utilizando el modelo EQTransformer. Este modelo ha demostrado ser efectivo en la clasificación de eventos sísmicos, superando a los métodos tradicionales en términos de precisión y adaptabilidad. Los resultados obtenidos indican que EQTransformer no solo mejora la detección de la onda P, sino que también es capaz de manejar eventos sísmicos complejos con mayor eficacia.

\section{Trabajos Futuros}
Para futuras investigaciones, se sugiere explorar las siguientes áreas:
\begin{itemize}
    \item **Optimización del modelo**: Mejorar la arquitectura del EQTransformer para reducir el tiempo de entrenamiento y aumentar la eficiencia en dispositivos con recursos limitados.
    \item **Integración de técnicas avanzadas de preprocesamiento**: Implementar técnicas de filtrado y normalización de datos para mejorar la calidad de los sismogramas de entrada.
    \item **Exploración de modelos híbridos**: Combinar el EQTransformer con otros modelos de aprendizaje profundo para mejorar la detección de fases sísmicas en condiciones adversas.
    \item **Aplicación en tiempo real**: Desarrollar un sistema de alerta temprana basado en el modelo EQTransformer que pueda integrarse con redes sísmicas existentes y proporcionar alertas en tiempo real.
\end{itemize}
